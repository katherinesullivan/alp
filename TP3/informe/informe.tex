\documentclass[11pt]{article}
\usepackage[utf8]{inputenc}
\usepackage{amsfonts}
\usepackage{natbib}
\usepackage{graphicx}
\usepackage{amsmath}
\usepackage{amssymb}
\usepackage{mathrsfs} % Cursive font
\usepackage{graphicx}
\usepackage{ragged2e}
\usepackage{fancyhdr}
\usepackage{nameref}
\usepackage{wrapfig}
\usepackage{ebproof}

% to set spacing between lines
\usepackage{setspace}

% to rotate content 90 degrees
\usepackage{lscape}

% to define deriving trees
% \usepackage{ebproof}
\usepackage{prftree}

% defining command for the curly arrow
\newcommand{\curly}{\mathrel{\leadsto}}


\usepackage{mathtools}
\usepackage{xparse} \DeclarePairedDelimiterX{\Iintv}[1]{\llbracket}{\rrbracket}{\iintvargs{#1}}
\NewDocumentCommand{\iintvargs}{>{\SplitArgument{1}{,}}m}
{\iintvargsaux#1}
\NewDocumentCommand{\iintvargsaux}{mm} {#1\mkern1.5mu,\mkern1.5mu#2}

\makeatletter
\newcommand*{\currentname}{\@currentlabelname}
\makeatother

\usepackage[a4paper,hmargin=1in, vmargin=1.4in,footskip=0.25in]{geometry}


%\addtolength{\hoffset}{-1cm}
%\addtolength{\hoffset}{-2.5cm}
%\addtolength{\voffset}{-2.5cm}
\addtolength{\textwidth}{0.2cm}
%\addtolength{\textheight}{2cm}
\setlength{\parskip}{8pt}
\setlength{\parindent}{0.5cm}
\linespread{1.5}

\pagestyle{fancy}
\fancyhf{}
\rhead{TP3 - Cipullo, Sullivan}
\lhead{Análisis de Lenguajes de Programación}
\rfoot{\vspace{1cm} \thepage}

\renewcommand*\contentsname{\LARGE Índice}

\begin{document}

\begin{titlepage}
    \begin{center}
        \vfill
        \vfill
            \vspace{0.7cm}
            \noindent\textbf{\Huge Trabajo Práctico 3}\par
            \noindent\textbf{\Huge Análisis de Lenguajes de Programación}\par
            \vspace{.5cm}
        \vfill
        \noindent \textbf{\huge Alumnas:}\par
        \vspace{.5cm}
        \noindent \textbf{\Large Cipullo, Inés}\par
        \noindent \textbf{\Large Sullivan, Katherine}\par
 
        \vfill
        \large Universidad Nacional de Rosario \par
        \noindent\large 2021
    \end{center}
\end{titlepage}
\ \par


% Setting the spacing between lines
\setstretch{0}


\section*{Ejercicio 1}

Tomemos $\Gamma = \{x: E \to E \to E, \ y: E \to E, \ z: E\}$

\vspace{0.3cm}

\hspace{-0.95cm}\begin{prooftree}
    \hypo{x: E \to E \to E \in \Gamma}
    \infer1[(T-VAR)]{\Gamma \vdash x: E \to E \to E}
    \hypo{z: E \in \Gamma}
    \infer1[(T-VAR)]{\Gamma \vdash z: E}
    \infer2[(T-APP)]{\Gamma \vdash (x\ z) : E \to E}
    
    \hypo{y: E \to E \in \Gamma}
    \infer1[(T-VAR)]{\Gamma \vdash y: E \to E}
    \hypo{z: E \in \Gamma}
    \infer1[(T-VAR)]{\Gamma \vdash z: E}
    \infer2[(T-APP)]{\Gamma \vdash (y\ z): E}
    
    \infer2[(T-APP)]{\Gamma \vdash (x\ z) \ (y \ z): E}
    
    \infer1[(T-ABS)]{\{x: E \to E \to E, \ y: E \to E\} \vdash \lambda z: E \cdot (x\ z) \ (y \ z) : E \to E}
    
    \infer1[(T-ABS)]{\{x: E \to E \to E \} \vdash \lambda y: E \to E \cdot \lambda z: E \cdot (x \ z) \ (y \ z): (E \to E) \to E \to E}
    
    \infer1[(T-ABS)]{\vdash \lambda x: E \to E \to E \cdot \lambda y: E \to E \cdot \lambda z: E \cdot (x \ z) \ (y \ z): (E \to E \to E) \to (E \to E) \to E \to E}
    
\end{prooftree}

\section*{Ejercicio 2}

La función \textit{infer} retorna un valor de tipo \textit{Either String Type} para poder retornar el tipo deseado en caso de que se haya podido realizar una inferencia de manera correcta, o una string que explique el error en caso de encontrarse con uno. 

La función $(>>=)$ toma un dato del tipo \textit{Either String Type} y una función que dado un \textit{Type} devuelve un \textit{Either String Type}. Nuestra función chequea si el dato pasado es un \textit{String} (en nuestra implementación entendámoslo como un error), si es el caso, no hace nada con este dato (simplemente propaga el error), en cambio, si el dato pasado es un \textit{Type} le aplica la función pasada a este y $(>>=)$ devuelve ese resultado.

\end{document}