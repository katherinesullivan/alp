\documentclass[11pt]{article}
\usepackage[utf8]{inputenc}
\usepackage{amsfonts}
\usepackage{natbib}
\usepackage{graphicx}
\usepackage{amsmath}
\usepackage{amssymb}
\usepackage{mathrsfs} % Cursive font
\usepackage{graphicx}
\usepackage{ragged2e}
\usepackage{fancyhdr}
\usepackage{nameref}
\usepackage{wrapfig}
%\usepackage{ebproof}

% to set spacing between lines
\usepackage{setspace}

% to rotate content 90 degrees
\usepackage{lscape}

% to define deriving trees
\usepackage{ebproof}
% \usepackage{prftree}

% defining command for the curly arrow
\newcommand{\curly}{\mathrel{\leadsto}}


\usepackage{mathtools}
\usepackage{xparse} \DeclarePairedDelimiterX{\Iintv}[1]{\llbracket}{\rrbracket}{\iintvargs{#1}}
\NewDocumentCommand{\iintvargs}{>{\SplitArgument{1}{,}}m}
{\iintvargsaux#1}
\NewDocumentCommand{\iintvargsaux}{mm} {#1\mkern1.5mu,\mkern1.5mu#2}

\makeatletter
\newcommand*{\currentname}{\@currentlabelname}
\makeatother

\usepackage[a4paper,hmargin=1in, vmargin=1.4in,footskip=0.25in]{geometry}


%\addtolength{\hoffset}{-1cm}
%\addtolength{\hoffset}{-2.5cm}
%\addtolength{\voffset}{-2.5cm}
\addtolength{\textwidth}{0.2cm}
%\addtolength{\textheight}{2cm}
\setlength{\parskip}{8pt}
\setlength{\parindent}{0.5cm}
\linespread{1.5}

\pagestyle{fancy}
\fancyhf{}
\rhead{TP4 - Cipullo, Sullivan}
\lhead{Análisis de Lenguajes de Programación}
\rfoot{\vspace{1cm} \thepage}

\renewcommand*\contentsname{\LARGE Índice}

\begin{document}

\begin{titlepage}
    \begin{center}
        \vfill
        \vfill
            \vspace{0.7cm}
            \noindent\textbf{\Huge Trabajo Práctico 4}\par
            \noindent\textbf{\Huge Análisis de Lenguajes de Programación}\par
            \vspace{.5cm}
        \vfill
        \noindent \textbf{\huge Alumnas:}\par
        \vspace{.5cm}
        \noindent \textbf{\Large Cipullo, Inés}\par
        \noindent \textbf{\Large Sullivan, Katherine}\par
 
        \vfill
        \large Universidad Nacional de Rosario \par
        \noindent\large 2021
    \end{center}
\end{titlepage}
\ \par


% Setting the spacing between lines
\setstretch{0}

\section*{Semántica de While}
Enunciamos la semántica del comando \verb|while|:

\begin{center}
    \begin{prooftree}
        \hypo{}
        \infer1[WHILE]{\langle \textbf{while} \ b \ c, \sigma \rangle \rightsquigarrow \langle \textbf{if} \ b \ \textbf{then} \ c; \textbf{while} \ b \ c \ \textbf{else skip}, \sigma \rangle}
    \end{prooftree}
\end{center}


\section*{Ejercicio 1.a}

Debemos probar que \verb|State| es una mónada. Para eso, verificaremos que se cumplan las tres propiedades monádicas: 

% $\mathrel{\gg \! \! =}$ mathrel permite ponerlos juntitos y cada vez que le tiras un \! acerca más el primero al segundo

\begin{itemize}
    \item (monad.1): $return\ x \mathrel{\gg \! \! =} f = f x$
    \item (monad.2): $t \mathrel{\gg \! \! =} return = t$
    \item (monad.3): $(t \mathrel{\gg \! \! =} f) \mathrel{\gg \! \! =} g = t \mathrel{\gg \! \! =} (\lambda x \to f x \mathrel{\gg \! \! =} g))$
\end{itemize} 

\textbf{monad.1}

Debemos probar que 

$$return\ x \mathrel{\gg \! \! =} f = f\ x$$

Veamos que 
\vspace{2mm}

\begin{align*}
    &return\ x \mathrel{\gg \! \! =} f\\
    &= \{def\ \mathrel{\gg \! \! =}\} \\
    &State\ (\lambda s \to let\ (v :\mathrel{!}: s') = runState\ (return\ x)\ s\ in\ runState\ (f\ v)\ s') \\
    &= \{def\ return\} \\ 
    &State\ (\lambda s \to let\ (v :\mathrel{!}: s') = runState\ (State\ (\lambda s_1 \to let\ (x :\mathrel{!}: s_1)))\ s\ in\ runState\ (f\ v)\ s') \\
    &= \{def\ runState\} \\
    &State\ (\lambda s \to let\ (v :\mathrel{!}: s') = (x :\mathrel{!}: s)\ in \ runState\ (f\ v)\ s')\\
    &= \{def\ let\} \\
    &State\ (\lambda s \to runState\ (f\ x)\ s)) \\
    &= \{def\ runState\ y\ def\ (f\ x)\ (*)\} \\
    &State\ (\lambda s \to g(s))\\
    &= \{def\ (f\ x)\}\\
    &f\ x
\end{align*}
\vspace{2mm}

$(*)$ Como $f\ x$ es de tipo $State\ b$, lo podemos definir como $f\ x = State\ (\lambda t \to g(t))$, donde $g$ es una función genérica de tipo $a \to b$, donde $a$ es el tipo de $x$.

\vspace{2mm}

Por lo que vale (monad.1).
\vspace{5cm}

\textbf{monad.2}

Debemos probar que

$$t \mathrel{\gg \! \! =} return = t$$

Veamos que 
\vspace{2mm}

\begin{align*}
    &t \mathrel{\gg \! \! =} return\\
    &= \{def\ \mathrel{\gg \! \! =}\} \\
    &State\ (\lambda s \to let\ (v :\mathrel{!}: s') = runState\ t\ s\ in\ runState\ (return\ v)\ s')\\
    &=\{def\ return\} \\
    &State\ (\lambda s \to let\ (v :\mathrel{!}: s') = runState\ t\ s\ in\ runState\ (State\ (\lambda s_1 \to (v :\mathrel{!}: s_1)))\ s') \\
    &=\{def\ runState\} \\
    &State\ (\lambda s \to let\ (v :\mathrel{!}: s') = runState\ t\ s\ in\ (v :\mathrel{!}: s')) \\
    &= \{def\ let\}\\
    &State\ (\lambda s \to runState\ t\ s) \\
    &= \{def\ t\ (**)\}\\
    &State\ (\lambda s \to runState\ (State\ (\lambda y \to h(y)))\ s) \\
    &= \{def\ runState\} \\
    &State\ (\lambda s \to h(s)) \\
    &= \{def\ t\} \\
    &t 
\end{align*}
\vspace{2mm}

$(**)$ Como $t$ es de tipo $State\ a$, lo podemos definir como $t\ = State\ (\lambda y \to h(y))$

\vspace{2mm}

Por lo que vale (monad.2).
\vspace{12cm}

\textbf{monad.3}

Debemos probar que 

$$(t \mathrel{\gg \! \! =} f) \mathrel{\gg \! \! =} g = t \mathrel{\gg \! \! =} (\lambda x \to f\ x \mathrel{\gg \! \! =} g))$$

Veamos que 
\vspace{2mm}

\begin{align*}
    &t \mathrel{\gg \! \! =} (\lambda x \to f\ x \mathrel{\gg \! \! =} g)) \\
    &= \{def\ \mathrel{\gg \! \! =} \}\\
    &State\ (\lambda s \to let\ (v :\mathrel{!}: s') = runState\ t\ s\ \\
    &\ \ \ \ \ \ \ \ \ \ \ \ \ \ \ \ \ in\ runState\ ((\lambda x \to State\ (\lambda p \to let\ (r :\mathrel{!}: p') = runState\ (f\ x)\ p\ \\
    &\ \ \ \ \ \ \ \ \ \ \ \ \ \ \ \ \ \ \ \ \ \ \ \ \ \ \ \ \ \ \ \ \ \ \ \ \ \ \ \ \ \ \ \ \ \ \ \ \ \ \ \ \ \ \ \ \ \ \ \ \ \ in\ runState\ (g\ r)\ p'))\ v)\ s') \\ 
    &= \{def\ aplicacion\ (\lambda x)\} \\
    &State\ (\lambda s \to let\ (v :\mathrel{!}: s') = runState\ t\ s\ \\
    &\ \ \ \ \ \ \ \ \ \ \ \ \ \ \ \ \ in\ runState\ (State\ (\lambda p \to let\ (r :\mathrel{!}: p') = runState\ (f\ v)\ p\ \\
    &\ \ \ \ \ \ \ \ \ \ \ \ \ \ \ \ \ \ \ \ \ \ \ \ \ \ \ \ \ \ \ \ \ \ \ \ \ \ \ \ \ \ \ \ \ \ \ \ \ \ \ \ in\ runState\ (g\ r)\ p'))\ s') \\ 
    &= \{def\ runState\} \\
    &State\ (\lambda s \to let\ (v :\mathrel{!}: s') = runState\ t\ s\ \\
    &\ \ \ \ \ \ \ \ \ \ \ \ \ \ \ \ \ in\ (\lambda p \to let\ (r :\mathrel{!}: p') = runState\ (f\ v)\ p\ 
                                              in\ runState\ (g\ r)\ p')\ s') \\ 
    &= \{def\ aplicacion\ (\lambda p)\} \\
    &State\ (\lambda s \to let\ (v :\mathrel{!}: s') = runState\ t\ s\ \\
    &\ \ \ \ \ \ \ \ \ \ \ \ \ \ \ \ \ in\ let\ (r :\mathrel{!}: p') = runState\ (f\ v)\ s'\ 
                                           in\ runState\ (g\ r)\ p') \\ 
    &= \{def\ let\} \\
    &State\ (\lambda s \to let\ (v :\mathrel{!}: s') = runState\ t\ s\ \\
    &\ \ \ \ \ \ \ \ \ \ \ \ \ \ \ \ \ \ \ \ \ \ (r :\mathrel{!}: p') = runState\ (f\ v)\ s'\\
    &\ \ \ \ \ \ \ \ \ \ \ \ \ \ \ \ \ in\ runState\ (g\ r)\ p') \\ 
    &= \{def\ let\} \\
    &State\ (\lambda s \to let\ (r :\mathrel{!}: p') = runState\ (State\ (\lambda s'' \to let\ (v :\mathrel{!}: s') = runState\ t\ s''\ \\
    &\ \ \ \ \ \ \ \ \ \ \ \ \ \ \ \ \ \ \ \ \ \ \ \ \ \ \ \ \ \ \ \ \ \ \ \ \ \ \ \ \ \ \ \ \ \ \ \ \ \ \ \ \ \ \ \ \ \ \ \ \ \ \ \ \ \ \ \ in\ runState\ (f\ v)\ s'))\ s \\
    &\ \ \ \ \ \ \ \ \ \ \ \ \ \ \ \ \ in\ runState\ (g\ r)\ p') \\ 
    &= \{def\ \mathrel{\gg \! \! =} \}\\
    &State\ (\lambda s \to let\ (r :\mathrel{!}: p') = runState\ (t \mathrel{\gg \! \! =} f)\ s\ in\ runState\ (g\ r)\ p')\\ 
    &= \{def\ \mathrel{\gg \! \! =} \}\\
    &(t \mathrel{\gg \! \! =} f) \mathrel{\gg \! \! =} g\\
\end{align*}

\vspace{2mm}

Por lo que vale (monad.3).

\vspace{10cm}

% \section*{Semántica de While}
% Enunciamos la semántica del comando \verb|while|:

% \begin{center}
%     \begin{prooftree}
%         \hypo{}
%         \infer1[WHILE]{\langle \textbf{while} \ b \ c, \sigma \rangle \rightsquigarrow \langle \textbf{if} \ b \ \textbf{then} \ c; \textbf{while} \ b \ c \ \textbf{else skip}, \sigma \rangle}
%     \end{prooftree}
% \end{center}

\end{document}