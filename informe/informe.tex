\documentclass[11pt]{article}
\usepackage[utf8]{inputenc}
\usepackage{amsfonts}
\usepackage{natbib}
\usepackage{graphicx}
\usepackage{amsmath}
\usepackage{amssymb}
\usepackage{mathrsfs} % Cursive font
\usepackage{graphicx}
\usepackage{ragged2e}
\usepackage{fancyhdr}
\usepackage{nameref}
\usepackage{wrapfig}
\usepackage{ebproof}

\usepackage{listings}
% for haskell style
\usepackage{xcolor}
\definecolor{mGreen}{rgb}{0,0.6,0}
\definecolor{mGray}{rgb}{0.5,0.5,0.5}
\definecolor{mPurple}{rgb}{0.58,0,0.82}
\definecolor{backgroundColour}{rgb}{0.95,0.95,0.92}

\lstdefinestyle{HaskellStyle}{
    backgroundcolor=\color{backgroundColour},   
    commentstyle=\color{mGreen},
    keywordstyle=\color{magenta},
    numberstyle=\tiny\color{mGray},
    stringstyle=\color{mPurple},
    basicstyle=\footnotesize,
    breakatwhitespace=false,         
    breaklines=true,                 
    captionpos=b,                    
    keepspaces=true,                 
    numbers=left,                    
    numbersep=5pt,                  
    showspaces=false,                
    showstringspaces=false,
    showtabs=false,                  
    tabsize=2,
    language=Haskell
}

% to set spacing between lines
\usepackage{setspace}

% to rotate content 90 degrees
\usepackage{lscape}

% to define deriving trees
% \usepackage{ebproof}
\usepackage{prftree}

% defining command for the curly arrow
\newcommand{\curly}{\mathrel{\leadsto}}


\usepackage{mathtools}
\usepackage{xparse} \DeclarePairedDelimiterX{\Iintv}[1]{\llbracket}{\rrbracket}{\iintvargs{#1}}
\NewDocumentCommand{\iintvargs}{>{\SplitArgument{1}{,}}m}
{\iintvargsaux#1}
\NewDocumentCommand{\iintvargsaux}{mm} {#1\mkern1.5mu,\mkern1.5mu#2}

\makeatletter
\newcommand*{\currentname}{\@currentlabelname}
\makeatother

\usepackage[a4paper,hmargin=1in, vmargin=1.4in,footskip=0.25in]{geometry}


%\addtolength{\hoffset}{-1cm}
%\addtolength{\hoffset}{-2.5cm}
%\addtolength{\voffset}{-2.5cm}
\addtolength{\textwidth}{0.2cm}
%\addtolength{\textheight}{2cm}
\setlength{\parskip}{8pt}
\setlength{\parindent}{0.5cm}
\linespread{1.5}

\pagestyle{fancy}
\fancyhf{}
\rhead{TP1 - Cipullo, Sullivan}
\lhead{Análisis de Lenguajes de Programación}
\rfoot{\vspace{1cm} \thepage}

\renewcommand*\contentsname{\LARGE Índice}

\begin{document}

\begin{titlepage}
    \begin{center}
        \vfill
        \vfill
            \vspace{0.7cm}
            \noindent\textbf{\Huge Trabajo Práctico 1}\par
            \noindent\textbf{\Huge Análisis de Lenguajes de Programación}\par
            \vspace{.5cm}
        \vfill
        \noindent \textbf{\huge Alumnas:}\par
        \vspace{.5cm}
        \noindent \textbf{\Large Cipullo, Inés}\par
        \noindent \textbf{\Large Sullivan, Katherine}\par
 
        \vfill
        \large Universidad Nacional de Rosario \par
        \noindent\large 2021
    \end{center}
\end{titlepage}
\ \par


% Setting the spacing between lines
\setstretch{0}


\section*{Ejercicio 1}

\subsection*{Gramática Abstracta}

\begin{align*}
    intexp\ ::=\ & nat\ |\ var\ |\ -_u\ intexp \\
            |\ \ & intexp\ +\ intexp \\ 
            |\ \ & intexp\ -_b\ intexp \\ 
            |\ \ & intexp\ \times\ intexp \\ 
            |\ \ & intexp\ \div\ intexp \\ 
            |\ \ & var\ =\ intexp \\
            |\ \ & intexp\ ,\ intexp \\
    boolexp\ ::=\ & \textbf{true}\ |\ \textbf{false} \\
             |\ \ & intexp\ ==\ intexp \\
             |\ \ & intexp\ \ne\ intexp \\
             |\ \ & intexp\ <\ intexp \\
             |\ \ & intexp\ >\ intexp \\
             |\ \ & boolexp\ \wedge\ boolexp \\
             |\ \ & boolexp\ \vee\ boolexp \\
             |\ \ & \neg\ boolexp \\
    comm\ :=\ & \textbf{skip} \\
          |\ \ & var = intexp \\
          |\ \ & comm;\ comm \\
          |\ \ & \textbf{if}\ boolexp\ \textbf{then}\ comm\ \textbf{else}\ comm \\
          |\ \ & \textbf{repeat}\ comm\ \textbf{until}\ boolexp
\end{align*}

\subsection*{Gramática Concreta}

Estan mal algunos simbolos

\begin{align*}
    digit\ ::=\ & \textbf{`0'}\ |\ \textbf{`1'}\ |\ \dots\ |\ \textbf{`9'} \\
    letter\ ::=\ & \textbf{`a'}\ |\ \dots\ |\ \textbf{`Z'} \\
    nat\ ::=\ & digit\ |\ digit\ nat \\
    var\ ::=\ & letter\ |\ letter\ var \\
    intexp\ ::=\ & nat \\
            |\ \ & var \\
            |\ \ & \textbf{`-'}\ intexp \\
            |\ \ & intexp\ \textbf{`+'}\ intexp \\ 
            |\ \ & intexp\ \textbf{`-'}\ intexp \\ 
            |\ \ & intexp\ \textbf{`*'}\ intexp \\ 
            |\ \ & intexp\ \textbf{`/'}\ intexp \\ 
            |\ \ & \textbf{`('}\ intexp\ \textbf{`)'} \\
            |\ \ & var\ \textbf{`='}\ intexp \\
            |\ \ & intexp\ \textbf{`,'}\ intexp \\
    boolexp\ ::=\ & \textbf{`true'}\ |\ \textbf{`false'} \\
             |\ \ & intexp\ \textbf{`=='}\ intexp \\
             |\ \ & intexp\ \textbf{`!='}\ intexp \\
             |\ \ & intexp\ \textbf{`<'}\ intexp \\
             |\ \ & intexp\ \textbf{`>'}\ intexp \\
             |\ \ & boolexp\ \textbf{`\&\&'}\ boolexp \\
             |\ \ & boolexp\ \textbf{`||'}\ boolexp \\
             |\ \ & \textbf{`!'}\ boolexp \\
             |\ \ & \textbf{`('}\ boolexp\ \textbf{`)'} \\
    comm\ :=\ & \textbf{\emph{skip}} \\
          |\ \ & var\ \textbf{`='}\ intexp \\
          |\ \ & comm\ \textbf{`;'}\ comm \\
          |\ \ & \textbf{`if'}\ boolexp\ \textbf{`\{'}\ comm\ \textbf{`\}'} \\
          |\ \ & \textbf{`if'}\ boolexp\ \textbf{`\{'}\ comm\ \textbf{`\}' `else' `\{'}\ comm \\
          |\ \ & \textbf{`repeat'}\ comm\ \textbf{`until'}\ boolexp\ \textbf{`end'}
\end{align*}


\section*{Ejercicio 6}

Habria que enunciar las nuevas reglas (ejercicio 4) y cambiar el nombre aca, y enunciar las reglas de la clausura transitiva.

\begin{landscape}


\textbf{A:}

\begin{displaymath}
    \prftree[r]{$\mathrm{T_1}$}{
        \prftree[r]{$\mathrm{SEQ_1}$}
        {\langle \textbf{skip}; \textbf{repeat}\ x=x-y\ \textbf{until}\ x==0,\ [[\sigma|x:1]|y:1] \rangle \curly \langle \textbf{repeat}\ x=x-y\ \textbf{until}\ x==0,\ [[\sigma|x:1]|y:1] \rangle}
    }
    {\langle \textbf{skip}; \textbf{repeat}\ x=x-y\ \textbf{until}\ x==0,\ [[\sigma|x:1]|y:1] \rangle \curly^* \langle \textbf{repeat}\ x=x-y\ \textbf{until}\ x==0,\ [[\sigma|x:1]|y:1] \rangle}
\end{displaymath}

\textbf{B:}

\begin{displaymath}
    \prftree[r]{$\mathrm{T_1}$}{
        \prftree[r]{$\mathrm{REPEAT}$}
        {\langle \textbf{repeat}\ x=x-y\ \textbf{until}\ x==0,\ [[\sigma|x:1]|y:1] \rangle \curly \langle x=x-y; \textbf{if}\ x==0\ \textbf{then\ skip\ else\ repeat}\ x=x-y\ \textbf{until}\ x==0,\ [[\sigma|x:1]|y:1] \rangle}
    }
    {\langle \textbf{repeat}\ x=x-y\ \textbf{until}\ x==0,\ [[\sigma|x:1]|y:1] \rangle \curly^* \langle x=x-y; \textbf{if}\ x==0\ \textbf{then\ skip\ else\ repeat}\ x=x-y\ \textbf{until}\ x==0,\ [[\sigma|x:1]|y:1] \rangle}
\end{displaymath}

\textbf{C:}

\begin{displaymath}
    \prftree[r]{$\mathrm{T_1}$}{
        \prftree[r]{$\mathrm{SEQ_2}$}{
            \prftree[r]{$\mathrm{ASS}$}{
                \prftree[r]{$\mathrm{MINUS}$}{
                    \prftree[r]{$\mathrm{VAR}$}
                    {\langle x,\ [[\sigma|x:1]|y:1] \rangle \Downarrow_{exp} \langle \textbf{1},\ [[\sigma|x:1]|y:1] \rangle}
                }{
                    \prftree[r]{$\mathrm{VAR}$}
                    {\langle y,\ [[\sigma|x:1]|y:1] \rangle \Downarrow_{exp} \langle \textbf{1},\ [[\sigma|x:1]|y:1] \rangle}  
                }
                {\langle x-y,\ [[\sigma|x:1]|y:1] \rangle \Downarrow_{exp} \langle \textbf{0},\ [[\sigma|x:1]|y:1] \rangle}
            }
            {\langle x=x-y,\ [[\sigma|x:1]|y:1] \rangle \curly \langle \textbf{skip},\ [[\sigma|x:0]|y:1] \rangle}
        }
        {\langle x=x-y; \textbf{if}\ x==0\ \textbf{then\ skip\ else\ repeat}\ x=x-y\ \textbf{until}\ x==0,\ [[\sigma|x:1]|y:1] \rangle \curly \langle \textbf{skip}; \textbf{if}\ x==0\ \textbf{then\ skip\ else\ repeat}\ x=x-y\ \textbf{until}\ x==0,\ [[\sigma|x:0]|y:1] \rangle}
    }
    {\langle x=x-y; \textbf{if}\ x==0\ \textbf{then\ skip\ else\ repeat}\ x=x-y\ \textbf{until}\ x==0,\ [[\sigma|x:1]|y:1] \rangle \curly^* \langle \textbf{skip}; \textbf{if}\ x==0\ \textbf{then\ skip\ else\ repeat}\ x=x-y\ \textbf{until}\ x==0,\ [[\sigma|x:0]|y:1] \rangle}
\end{displaymath}

\textbf{D:}

\begin{displaymath}
    \prftree[r]{$\mathrm{T_1}$}{
        \prftree[r]{$\mathrm{SEQ_2}$}{
            \prftree[r]{$\mathrm{ASS}$}{
                \prftree[r]{$\mathrm{NASS}$}{
                    \prftree[r]{$\mathrm{NVAL}$}
                    {\langle 1,\ [[\sigma|x:2]|y:2] \rangle \Downarrow_{exp} \langle \textbf{1} ,\ [[\sigma|x:2]|y:2] \rangle}
                }
                {\langle y=1,\ [[\sigma|x:2]|y:2] \rangle \Downarrow_{exp} \langle \textbf{1} ,\ [[\sigma|x:2]|y:1] \rangle}
            }
            {\langle x=y=1,\ [[\sigma|x:2]|y:2] \rangle \curly \langle \textbf{skip} ,\ [[\sigma|x:1]|y:1] \rangle}
        }
        {\langle x=y=1; \textbf{repeat}\ x=x-y\ \textbf{until}\ x==0,\ [[\sigma|x:2]|y:2] \rangle \curly \langle \textbf{skip}; \textbf{repeat}\ x=x-y\ \textbf{until}\ x==0,\ [[\sigma|x:1]|y:1] \rangle}
    }
    {\langle x=y=1; \textbf{repeat}\ x=x-y\ \textbf{until}\ x==0,\ [[\sigma|x:2]|y:2] \rangle \curly^* \langle \textbf{skip}; \textbf{repeat}\ x=x-y\ \textbf{until}\ x==0,\ [[\sigma|x:1]|y:1] \rangle}
\end{displaymath}

\textbf{E:}

\begin{displaymath}
    \prftree[r]{$\mathrm{T_1}$}{
        \prftree[r]{$\mathrm{SEQ_1}$}
        {\langle \textbf{skip}; \textbf{if}\ x==0\ \textbf{then\ skip\ else\ repeat}\ x=x-y\ \textbf{until}\ x==0,\ [[\sigma|x:0]|y:1] \rangle \curly \langle \textbf{if}\ x==0\ \textbf{then\ skip\ else\ repeat}\ x=x-y\ \textbf{until}\ x==0,\ [[\sigma|x:0]|y:1] \rangle}
    }
    {\langle \textbf{skip}; \textbf{if}\ x==0\ \textbf{then\ skip\ else\ repeat}\ x=x-y\ \textbf{until}\ x==0,\ [[\sigma|x:0]|y:1] \rangle \curly^* \langle \textbf{if}\ x==0\ \textbf{then\ skip\ else\ repeat}\ x=x-y\ \textbf{until}\ x==0,\ [[\sigma|x:0]|y:1] \rangle}
\end{displaymath}

\textbf{F:}

\begin{displaymath}
    \prftree[r]{$\mathrm{T_1}$}{
        \prftree[r]{$\mathrm{IF_1}$}{
            \prftree[r]{$\mathrm{EQ}$}{
                \prftree[r]{$\mathrm{VAR}$}
                {\langle x,\ [[\sigma|x:0]|y:1] \rangle \Downarrow_{exp} \langle \textbf{0},\ [[\sigma|x:0]|y:1] \rangle}
            }{
                \prftree[r]{$\mathrm{NVAL}$}
                {\langle 0,\ [[\sigma|x:0]|y:1] \rangle \Downarrow_{exp} \langle \textbf{0},\ [[\sigma|x:0]|y:1] \rangle}
            }
            {\langle x==0,\ [[\sigma|x:0]|y:1] \rangle \Downarrow_{exp} \langle \textbf{true},\ [[\sigma|x:0]|y:1] \rangle}
        }
        {\langle \textbf{if}\ x==0\ \textbf{then\ skip\ else\ repeat}\ x=x-y\ \textbf{until}\ x==0,\ [[\sigma|x:0]|y:1] \rangle \curly \langle \textbf{skip},\ [[\sigma|x:0]|y:1] \rangle}
    }
    {\langle \textbf{if}\ x==0\ \textbf{then\ skip\ else\ repeat}\ x=x-y\ \textbf{until}\ x==0,\ [[\sigma|x:0]|y:1] \rangle \curly^* \langle \textbf{skip},\ [[\sigma|x:0]|y:1] \rangle}
\end{displaymath}

\textbf{DEM:}

\begin{displaymath}
    \prftree[r]{$\mathrm{T_2}$}{
        \prftree[r]{$\mathrm{T_2}$}{
            \prftree[r]{$\mathrm{T_2}$}{
                \prftree[r]{$\mathrm{T_2}$}{
                    \prftree[r]{$\mathrm{T_2}$}{D}{A}
                    {\langle x=y=1; \textbf{repeat}\ x=x-y\ \textbf{until}\ x==0,\ [[\sigma|x:2]|y:2] \rangle \curly^* \langle \textbf{repeat}\ x=x-y\ \textbf{until}\ x==0,\ [[\sigma|x:1]|y:1] \rangle}
                }{B}
                {\langle x=y=1; \textbf{repeat}\ x=x-y\ \textbf{until}\ x==0,\ [[\sigma|x:2]|y:2] \rangle \curly^* \langle x=x-y; \textbf{if}\ x==0\ \textbf{then\ skip\ else\ repeat}\ x=x-y\ \textbf{until}\ x==0,\ [[\sigma|x:1]|y:1] \rangle}
            }{C}
            {\langle x=y=1; \textbf{repeat}\ x=x-y\ \textbf{until}\ x==0,\ [[\sigma|x:2]|y:2] \rangle \curly^* \langle \textbf{skip}; \textbf{if}\ x==0\ \textbf{then\ skip\ else\ repeat}\ x=x-y\ \textbf{until}\ x==0,\ [[\sigma|x:0]|y:1] \rangle}
        }{E}
        {\langle x=y=1; \textbf{repeat}\ x=x-y\ \textbf{until}\ x==0,\ [[\sigma|x:2]|y:2] \rangle \curly^* \langle \textbf{if}\ x==0\ \textbf{then\ skip\ else\ repeat}\ x=x-y\ \textbf{until}\ x==0,\ [[\sigma|x:0]|y:1] \rangle}
    }{F}
    {\langle x=y=1; \textbf{repeat}\ x=x-y\ \textbf{until}\ x==0,\ [[\sigma|x:2]|y:2] \rangle \curly^* \langle \textbf{skip},\ [[\sigma|x:0]|y:1] \rangle}
\end{displaymath}
\end{landscape}


\section*{Ejercicio 2}

\begin{lstlisting}[style = HaskellStyle]
    -- Expresiones, aritmeticas y booleanas
data Exp a where
  -- Expresiones enteras
  Const ::Int -> Exp Int
  Var ::Variable -> Exp Int
  UMinus ::Exp Int -> Exp Int
  Plus ::Exp Int -> Exp Int -> Exp Int
  Minus ::Exp Int -> Exp Int -> Exp Int
  Times ::Exp Int -> Exp Int -> Exp Int
  Div ::Exp Int -> Exp Int -> Exp Int
  EAssgn ::Variable -> Exp Int -> Exp Int
  ESeq ::Exp Int -> Exp Int -> Exp Int

  -- Expreseiones booleanas
  BTrue ::Exp Bool
  BFalse ::Exp Bool
  Lt ::Exp Int -> Exp Int -> Exp Bool
  Gt ::Exp Int -> Exp Int -> Exp Bool
  And ::Exp Bool -> Exp Bool -> Exp Bool
  Or ::Exp Bool -> Exp Bool -> Exp Bool
  Not ::Exp Bool -> Exp Bool
  Eq ::Exp Int -> Exp Int -> Exp Bool
  NEq ::Exp Int -> Exp Int -> Exp Bool
\end{lstlisting}

\section*{Ejercicio 4}

\subsection*{Sem\'antica Operacional Big-Step para Expresiones}
\vspace{0.5cm}

\begin{center}
\begin{prooftree}
    \hypo{}
    \infer1[NVAL]{\langle nv,\sigma \rangle \Downarrow_{exp} \langle \textbf{nv},\sigma \rangle }
\end{prooftree}
\hspace{1cm}
\begin{prooftree}
    \hypo{}
    \infer1[VAR]{\langle x,\sigma \rangle \Downarrow_{exp} \langle \sigma x,\sigma \rangle }
\end{prooftree}
\end{center}

\begin{center}
\begin{prooftree}
    \hypo{\langle e,\sigma \rangle \Downarrow_{exp} \langle n,\sigma' \rangle}
    \infer1[UMINUS]{\langle -_{u}e,\sigma \rangle \Downarrow_{exp} \langle -n, \sigma' \rangle}
\end{prooftree}
\hspace{1cm}
\begin{prooftree}
    \hypo{\langle e_0,\sigma \rangle \Downarrow_{exp} \langle n_0,\sigma' \rangle}
    \hypo{\langle e_1,\sigma' \rangle \Downarrow_{exp} \langle n_1,\sigma'' \rangle}
    \infer2[PLUS]{\langle e_0+e_1,\sigma \rangle \Downarrow_{exp} \langle n_0+n_1, \sigma'' \rangle}
\end{prooftree}
\end{center}

\begin{center}
\begin{prooftree}
    \hypo{\langle e_0,\sigma \rangle \Downarrow_{exp} \langle n_0,\sigma' \rangle}
    \hypo{\langle e_1,\sigma' \rangle \Downarrow_{exp} \langle n_1,\sigma'' \rangle}
    \infer2[BMINUS]{\langle e_0-e_1,\sigma \rangle \Downarrow_{exp} \langle n_0-n_1, \sigma'' \rangle}
\end{prooftree}
\end{center}

\begin{center}
\begin{prooftree}
    \hypo{\langle e_0,\sigma \rangle \Downarrow_{exp} \langle n_0,\sigma' \rangle}
    \hypo{\langle e_1,\sigma' \rangle \Downarrow_{exp} \langle n_1,\sigma'' \rangle}
    \infer2[MULT]{\langle e_0*e_1,\sigma \rangle \Downarrow_{exp} \langle n_0*n_1, \sigma'' \rangle}
\end{prooftree}
\end{center}

\begin{center}
\begin{prooftree}
    \hypo{\langle e_0,\sigma \rangle \Downarrow_{exp} \langle n_0,\sigma' \rangle}
    \hypo{\langle e_1,\sigma' \rangle \Downarrow_{exp} \langle n_1,\sigma'' \rangle}
    \hypo{n_1 \not = 0}
    \infer3[DIV]{\langle e_0\div e_1,\sigma \rangle \Downarrow_{exp} \langle n_0\div n_1, \sigma'' \rangle}
\end{prooftree}
\end{center}

\begin{center}
\begin{prooftree}
    \hypo{\langle e_0,\sigma \rangle \Downarrow_{exp} \langle n_0,\sigma' \rangle}
    \hypo{\langle e_1,\sigma' \rangle \Downarrow_{exp} \langle n_1,\sigma'' \rangle}
    \infer2[GT]{\langle e_0 > e_1,\sigma \rangle \Downarrow_{exp} \langle n_0 > n_1, \sigma'' \rangle}
\end{prooftree}
\end{center}

\begin{center}
\begin{prooftree}
    \hypo{\langle e_0,\sigma \rangle \Downarrow_{exp} \langle n_0,\sigma' \rangle}
    \hypo{\langle e_1,\sigma' \rangle \Downarrow_{exp} \langle n_1,\sigma'' \rangle}
    \infer2[LT]{\langle e_0 < e_1,\sigma \rangle \Downarrow_{exp} \langle n_0 < n_1, \sigma'' \rangle}
\end{prooftree}
\end{center}

\begin{center}
\begin{prooftree}
    \hypo{\langle e_0,\sigma \rangle \Downarrow_{exp} \langle n_0,\sigma' \rangle}
    \hypo{\langle e_1,\sigma' \rangle \Downarrow_{exp} \langle n_1,\sigma'' \rangle}
    \infer2[NOTEQ]{\langle e_0 \mathrel{\mathtt{!=}} e_1,\sigma \rangle \Downarrow_{exp} \langle n_0 \neq n_1, \sigma'' \rangle}
\end{prooftree}
\end{center}

\begin{center}
\begin{prooftree}
    \hypo{\langle e_0,\sigma \rangle \Downarrow_{exp} \langle n_0,\sigma' \rangle}
    \hypo{\langle e_1,\sigma' \rangle \Downarrow_{exp} \langle n_1,\sigma'' \rangle}
    \infer2[EQ]{\langle e_0 == e_1,\sigma \rangle \Downarrow_{exp} \langle n_0 = n_1, \sigma'' \rangle}
\end{prooftree}
\hspace{1cm}
\begin{prooftree}
    \hypo{}
    \infer1[BVAL]{\langle bv,\sigma \rangle \Downarrow_{exp} \langle bv,\sigma \rangle}
\end{prooftree}
\end{center}

\begin{center}
\begin{prooftree}
    \hypo{\langle p,\sigma \rangle \Downarrow_{exp} \langle b,\sigma' \rangle}
    \infer1[NOT]{\langle \lnot p,\sigma \rangle \Downarrow_{exp} \langle \lnot b, \sigma' \rangle}
\end{prooftree}
\hspace{1cm}
\begin{prooftree}
    \hypo{\langle p_0,\sigma \rangle \Downarrow_{exp} \langle b_0,\sigma' \rangle}
    \hypo{\langle p_1,\sigma' \rangle \Downarrow_{exp} \langle b_1,\sigma'' \rangle}
    \infer2[OR]{\langle e_0 \lor e_1,\sigma \rangle \Downarrow_{exp} \langle b_0 \lor b_1, \sigma'' \rangle}
\end{prooftree}
\end{center}

\begin{center}
\begin{prooftree}
    \hypo{\langle p_0,\sigma \rangle \Downarrow_{exp} \langle b_0,\sigma' \rangle}
    \hypo{\langle p_1,\sigma' \rangle \Downarrow_{exp} \langle b_1,\sigma'' \rangle}
    \infer2[AND]{\langle e_0 \land e_1,\sigma \rangle \Downarrow_{exp} \langle b_0 \land b_1, \sigma'' \rangle}
\end{prooftree}
\end{center}

Llamamos IASS a la asignaci\'on como expresi\'on y ISEQ a la secuencializaci\'on de expresiones con el operador \textbf{,}

\begin{center}
\begin{prooftree}
    \hypo{\langle e,\sigma \rangle \Downarrow_{exp} \langle n,\sigma' \rangle}
    \infer1[IASS]{\langle v=e, \sigma \rangle \Downarrow_{exp} \langle n, [\sigma' | v: n] \rangle}
\end{prooftree}
\end{center}

\begin{center}
\begin{prooftree}
    \hypo{\langle e_0,\sigma \rangle \Downarrow_{exp} \langle n_0,\sigma' \rangle}
    \hypo{\langle e_1,\sigma' \rangle \Downarrow_{exp} \langle n_1,\sigma'' \rangle}
    \infer2[ISEQ]{\langle e_0,e_1,\sigma \rangle \Downarrow_{exp} \langle n_1, \sigma'' \rangle}
\end{prooftree}
\end{center}

\section*{Ejercicio 5}

\subsection*{Determinismo de la relaci\'on de evaluaci\'on en un paso $\rightsquigarrow$}

\section*{Ejercicio 10}

\subsection*{Nueva gram\'atica abstracta}
for (intexp;boolexp;intexp) comm

\subsection*{Nueva sem\'antica operacional para comandos}

\begin{center}
\begin{prooftree}
    \hypo{\langle e,\sigma \rangle \Downarrow_{exp} \langle n,\sigma' \rangle}
    \infer1[ASS]{\langle v=e, \sigma \rangle \rightsquigarrow \langle \textbf{skip}, [\sigma' | v: n] \rangle}
\end{prooftree}

\begin{prooftree}
    \hypo{}
    \infer1[SEQ1]{\langle \textbf{skip}; c_1, \sigma \rangle \rightsquigarrow \langle c_1, \sigma \rangle}
\end{prooftree}

\begin{prooftree}
    \hypo{\langle c_0, \sigma \rangle \rightsquigarrow \langle c'_0, \sigma' \rangle}
    \infer1[SEQ2]{\langle \textbf{skip}; c_1, \sigma \rangle \rightsquigarrow \langle c'_0; c_1, \sigma' \rangle}
\end{prooftree}

\begin{prooftree}
    \hypo{\langle b, \sigma \rangle \Downarrow_{exp} \langle \textbf{true}, \sigma' \rangle}
    \infer1[IF_1]{\langle \textbf{if}\ b \ \textbf{then} \ c_0 \ \textbf{else} \ c_1, \sigma \rangle \rightsquigarrow \langle c_0, \sigma' \rangle}
\end{prooftree}
\hspace{1cm}
\begin{prooftree}
    \hypo{\langle b, \sigma \rangle \Downarrow_{exp} \langle \textbf{false}, \sigma' \rangle}
    \infer1[IF_2]{\langle \textbf{if}\ b \ \textbf{then} \ c_0 \ \textbf{else} \ c_1, \sigma \rangle \rightsquigarrow \langle c_1, \sigma' \rangle}
\end{prooftree}
\vspace{0.25cm}

\begin{prooftree}
    \hypo{}
    \infer1[REPEAT_1]{\langle \textbf{repeat} \ c \ \textbf{until} \ b, \sigma \rangle \rightsquigarrow \langle c; \textbf{if} \ b \ \textbf{then skip else repeat} \ b \ \textbf{until} \ c, \sigma \rangle}
\end{prooftree}

\begin{prooftree}
    \hypo{\langle e, \sigma \rangle \Downarrow_{exp} \langle n, \sigma' \rangle}
    \infer1[REPEAT_2]{\langle \textbf{repeat}\ (c; e) \ \textbf{until} \ b, \sigma \rangle \rightsquigarrow }
\end{prooftree}

\begin{prooftree}
    \hypo{\langle e_1, \sigma \rangle \Downarrow_{exp} \langle n_1, \sigma' \rangle}
    \infer1[FOR]{\langle \textbf{for} \ (e_1;e_2;e_3) \ c, \sigma \rangle \rightsquigarrow \langle \textbf{repeat}\ (c; e_3) \ \textbf{until} \ e_2, \sigma' \rangle}
\end{prooftree}
    
    
\end{center}

\end{document}